% Name: Till Uhlig
% Matrikelnummer: 211203809
% Studiengang: Informatik
% eMail: till.uhlig@student.uni-halle.de
%
\documentclass[10pt,a4paper,final,parskip]{scrartcl}
\usepackage[utf8]{inputenc}
\usepackage{lmodern}
\usepackage[babel, german=guillemets]{csquotes}
\usepackage[ngerman]{babel}
\usepackage{amsmath}
\usepackage{graphicx} % für die eingebundenen Bilder
\usepackage{algorithm} % für die Pseudocodes
\usepackage{algorithmic} % für die Pseudocodes
\usepackage{amsthm}
\usepackage{multicol}
\usepackage{wrapfig}
%\usepackage{thmbox}
\usepackage[pagestyles,extramarks]{titlesec} % habe ich für meine Kopfzeile benutzt
\usepackage{url} % für die URLs der Bibliographie
\usepackage{tabularx} % für die erweiterten tabular Funktionen
\usepackage{color} % dieses Package benötige ich für meinen \todo Befehl, um eine Farbige Box zu zeichnen
 \usepackage{amsmath}
 
\usepackage{listings,xcolor}
%\newtheorem[L]{thmL}{Definition}
%\newtheorem[M]{thmM}{Definition}
%\newtheorem[S]{thmS}{Definition}
\newtheorem{defi}{Definition}[section]
\numberwithin{equation}{section}

\floatname{algorithm}{Algorithmus}
\definecolor{codeGray}{RGB}{240,240,240}
\definecolor{codeBlack}{RGB}{0,0,0}
\definecolor{codeRed}{RGB}{221,0,0}
\definecolor{codeBlue}{rgb}{0,0,187}
\definecolor{codeYellow}{RGB}{255,128,0}
\definecolor{codeGreen}{RGB}{0,119,0}

% … und zuweisen
\lstset{%
    language=PHP,%
    %
    % Farben, diktengleiche Schrift
    backgroundcolor={\color{codeGray}},% 
    basicstyle={\small\ttfamily\color{codeGreen}},% 
    commentstyle={\color{codeYellow}},%
    keywordstyle={\color{codeBlue}},%
    stringstyle={\color{codeRed}},%
    identifierstyle={\color{codeBlue}},%
    %
    % Zeilenumbrüche aktivieren, Leerzeichen nicht hervorheben    
    breaklines=true,%
    showstringspaces=false,%
    % 
    % Listing-Caption unterhalb (bottom)
    captionpos=b,%
    % 
    % Listing einrahmen
    frame=single,%
    rulecolor={\color{codeBlack}},%
    % 
    % winzige Zeilennummern links
   % numbers=left,%
 %   numberstyle={\tiny\color{codeBlack}}%
}

\newcommand{\monthword}[1]{\ifcase#1\or Januar\or Februar\or M\"arz\or April\or Mai\or Juni\or July\or August\or September\or Oktober\or November\or Dezember\fi} 

% Grunddaten festlegen, für Titelseite
\title
{}
\author 
{}
\date{\vspace{12cm}\monthword{\the\month}\space\the\year}
\subtitle
{}


% hier werden Kopf- und Fusszeile festgelegt
\settitlemarks*{section,subsection}
\newpagestyle{seite}{
}

\newenvironment{php}
{\begin{minipage}{\textwidth}
\begin{lstlisting}}
{\end{lstlisting}
\end{minipage}}

\newenvironment{beispiel}
{\begin{minipage}{\textwidth}
Beispiel
\begin{lstlisting}}
{\end{lstlisting}
\end{minipage}}

\usepackage{chngcntr,tocloft}
\counterwithin*{figure}{section}
%\counterwithin*{figure}{subsection}
%\counterwithin*{figure}{subsubsection}

\renewcommand{\thefigure}{%
  %\ifnum\value{subsection}=0
    \thesection.\arabic{figure}%
  %\else
    %\ifnum\value{subsubsection}=0
     % \thesubsection%.
   %   \arabic{figure}%
    %%\else
    %%  \thesubsubsection.\arabic{figure}%
   % \fi
 % \fi
}


% diese Festlegungen werden für die \setlemma und \setdef benötigt
\newtheorem{theorem}{Theorem}[section]
\newtheorem{lemma}[theorem]{Lemma}
\newenvironment{definition}[1][Definition]{\begin{trivlist}
\item[\hskip \labelsep {\bfseries #1}]}{\end{trivlist}}
      
% zum erzeugen einer Lemmas oder einer Defintion      
\newcommand{\setdef}[2]{\begin{definition}[#1] #2 \end{definition}}
\newcommand{\setlemma}[2]{\begin{lemma}[#1] #2 \end{lemma}}

\newcommand{\abb}[1]{Abbildung \ref{#1}}

% Quellenangaben werden mit diesen Befehlen realisiert
\newcommand{\zitat}[1]{\tiny{\cite{#1}}\normalsize}
\newcommand{\Zzitat}[1]{\tiny{#1}\normalsize}

% zum erzeugen von "todo" Zeilen
\newcommand{\blau}[1]{\textcolor{blue}{#1}}
\newcommand{\todo}[1]{\colorbox{yellow}{\parbox{\textwidth}{Todo: \textbf{#1}}}}

\newenvironment{Bilder}
  {\par\raggedbottom\null\noindent\minipage{\textwidth}\centering}
  {\endminipage\vspace{0.7cm}}
  
\newcommand{\bild}[3]{
\begin{Bilder}
 \fbox{\includegraphics[width=\textwidth]{Images/#1}}
	\captionof{figure}{#2}
	\label{#3}
	\end{Bilder}
}

\newcommand{\wbild}[4]{%\textwidth
\begin{wrapfigure}{r}{#4\textwidth}
\centering
\fbox{\includegraphics[scale=.75]{Images/#1}}
	\captionof{figure}{#2}
	\label{#3}
\end{wrapfigure}
}

\newcommand{\dbild}[3]{%\textwidth
\begin{Bilder}
 \fbox{\includegraphics[scale=.5]{Images/#1}}
	%\captionof{figure}{#2}
	\label{#3}
	\end{Bilder}
}

\newcommand{\tbild}[3]{%\textwidth
\begin{Bilder}
 \includegraphics[scale=.5]{Images/#1}
	\captionof{figure}{#2}
	\label{#3}
	\end{Bilder}
}

\newcommand{\gbild}[3]{%
\begin{Bilder}
 \fbox{\includegraphics[width = \textwidth]{Images/#1}}
	\captionof{figure}{#2}
	\label{#3}
	\end{Bilder}
}

\newcommand{\pbild}[3]{%\textwidth
\begin{Bilder}
 \fbox{\includegraphics[scale=.50]{Images/#1}}
	\captionof{figure}{#2}
	\label{#3}
	\end{Bilder}
}

% Ich habe öfters den Fall gehabt, dass ich gerne ein Lemma und
% ein passendes Bild daneben darstellen wollte, darum habe ich
% mir diese Gleitumgebung eingerichtet.
\newcommand{\nebeneinander}[2]{
\begin{figure}[H]
  \begin{minipage}[b]{.5\linewidth}
    \centering
#1
  \end{minipage}
  \begin{minipage}[b]{.5\linewidth}
    \centering
#2
  \end{minipage}\hfill
\end{figure}
}


% hier beginnt unser Dokument
\begin{document}

% zunächst die Titelseite ausgeben
%\cleardoublepage
%\maketitle

% und unser Inhaltsverzeichnis
%\cleardoublepage
%\tableofcontents


%\cleardoublepage
% ab hier nutze ich die oben definierten Kopf- und Fusszeilen
\pagestyle{seite}
%\dbild{Formular_erstellen.png}{}{}
Sie müssen Ihre Gruppen in jeder Übungsserie erstellen/annehmen, denn Gruppenstrukturen werden zwischen den Übungsserien nicht automatisch übernommen.

\section{Gruppe erstellen}
Sie können eine Gruppe erstellen, indem Sie sich in die \blau{Gruppenverwaltung} begeben (in Ihrer \blau{Veranstaltungsübersicht} unter \blau{Gruppe verwalten}, wenn Gruppenbildung erlaubt ist).
Geben Sie das 5-stellige Kürzel ihres zukünftigen Gruppenmitgliedes unter \blau{Einladungen zum Beitritt versenden} ein und bestätigen Sie das Versenden der Einladung, über die Schaltfläche \blau{versenden} (siehe \abb{einladen}).
\tbild{2.png}{Gruppenmitglieder einladen}{einladen}

\tbild{3.png}{Einladung wurde erfolgreich verschickt}{positivEinladen}
Der Eingeladene muss nun die Einladung in seiner \blau{Gruppenverwaltung} annehmen, um den Vorgang abzuschließen.

\tbild{3f.png}{unbekanntes Kürzel}{negativEinladen}
Wenn Sie eine Fehlermeldung, wie in \abb{negativEinladen}, erhalten, sollten Sie das eingegebene Kürzel auf Korrektheit prüfen (eventuell klein schreiben).

\section{Gruppe beitreten}
Zum betreten einer Gruppe ist eine Einladung eines Gruppenführers erforderlich (siehe \blau{Gruppe erstellen}). Sie finden diese Einladungen in Ihrer \blau{Gruppenverwaltung} unter \blau{Einladungen zu anderen Gruppen}, wo Sie diese annehmen und ablehnen können (siehe \abb{beitreten}).

\tbild{4.png}{Einladung annehmen}{beitreten}

\tbild{9.png}{Einladung wurde erfolgreich angenommen}{positivBeitreten}

\section{Gruppe verlassen}
Sie können eine Gruppe auflösen (als Gruppenführer) oder verlassen (als Gruppenmitglied), indem Sie die Schaltfläche \blau{verlassen} in der \blau{Gruppenverwaltung} unter \blau{Gruppenübersicht} betätigen (siehe \abb{verlassen}).
 
\tbild{7.png}{Gruppe verlassen/auflösen}{verlassen}

\tbild{8.png}{Gruppe wurde verlassen}{Verlassen}
Sollten Sie eine Gruppen verlassen/aufgelöst haben, müssen Sie ihre Einsendungen über den \blau{Einsendungsverlauf} für die Korrektur auswählen oder Ihre Abgaben erneut einsenden (\blau{Aufgaben hochladen}), weil durch das Verlassen der Gruppe die Einsendungsauswahl verloren geht. 

\section{Einsendungen auswählen}
Als \blau{Gruppenführer}), können Sie in Ihrer \blau{Gruppenverwaltung}, im Bereich \blau{Gruppeneinsendungen verwalten}, festlegen welche Einsendung der Gruppenmitglieder für die Korrektur bestimmt werden sollen. Wählen Sie dazu die entsprechende Einsendung aus und bestätigen Sie die Änderung über \blau{speichern} (siehe \abb{selektieren}).
\tbild{5.png}{Einsendungen auswählen}{selektieren}
\tbild{6.png}{Einsendung wurde ausgewählt}{positivSelektieren}
Die ausgewählten Einsendungen werden allen Gruppenmitgliedern gleichermaßen auf deren \blau{Veranstaltungsübersicht} angezeigt.

\end{document}
% Name: Till Uhlig
% Matrikelnummer: 211203809
% Studiengang: Informatik
% eMail: till.uhlig@student.uni-halle.de
%
\documentclass[a4paper,final]{scrartcl}
\usepackage[utf8]{inputenc}
\usepackage{lmodern}
\usepackage[babel, german=guillemets]{csquotes}
\usepackage[ngerman]{babel}
\usepackage{amsmath}
\usepackage{graphicx} % für die eingebundenen Bilder
\usepackage{algorithm} % für die Pseudocodes
\usepackage{algorithmic} % für die Pseudocodes
\usepackage{amsthm}
\usepackage{multicol}
\usepackage{wrapfig}
%\usepackage{thmbox}
\usepackage[pagestyles,extramarks]{titlesec} % habe ich für meine Kopfzeile benutzt
\usepackage{url} % für die URLs der Bibliographie
\usepackage{tabularx} % für die erweiterten tabular Funktionen
\usepackage{color} % dieses Package benötige ich für meinen \todo Befehl, um eine Farbige Box zu zeichnen
 \usepackage{amsmath}
 
\usepackage{listings,xcolor}
%\newtheorem[L]{thmL}{Definition}
%\newtheorem[M]{thmM}{Definition}
%\newtheorem[S]{thmS}{Definition}
\newtheorem{defi}{Definition}[section]
\numberwithin{equation}{section}

\floatname{algorithm}{Algorithmus}
\definecolor{codeGray}{RGB}{240,240,240}
\definecolor{codeBlack}{RGB}{0,0,0}
\definecolor{codeRed}{RGB}{221,0,0}
\definecolor{codeBlue}{rgb}{0,0,187}
\definecolor{codeYellow}{RGB}{255,128,0}
\definecolor{codeGreen}{RGB}{0,119,0}

% … und zuweisen
\lstset{%
    language=PHP,%
    %
    % Farben, diktengleiche Schrift
    backgroundcolor={\color{codeGray}},% 
    basicstyle={\small\ttfamily\color{codeGreen}},% 
    commentstyle={\color{codeYellow}},%
    keywordstyle={\color{codeBlue}},%
    stringstyle={\color{codeRed}},%
    identifierstyle={\color{codeBlue}},%
    %
    % Zeilenumbrüche aktivieren, Leerzeichen nicht hervorheben    
    breaklines=true,%
    showstringspaces=false,%
    % 
    % Listing-Caption unterhalb (bottom)
    captionpos=b,%
    % 
    % Listing einrahmen
    frame=single,%
    rulecolor={\color{codeBlack}},%
    % 
    % winzige Zeilennummern links
   % numbers=left,%
 %   numberstyle={\tiny\color{codeBlack}}%
}

\newcommand{\monthword}[1]{\ifcase#1\or Januar\or Februar\or M\"arz\or April\or Mai\or Juni\or July\or August\or September\or Oktober\or November\or Dezember\fi} 

% Grunddaten festlegen, für Titelseite
\title
{}
\author 
{}
\date{\vspace{12cm}\monthword{\the\month}\space\the\year}
\subtitle
{}


% hier werden Kopf- und Fusszeile festgelegt
\settitlemarks*{section,subsection}
\newpagestyle{seite}{
}

\newenvironment{php}
{\begin{minipage}{\textwidth}
\begin{lstlisting}}
{\end{lstlisting}
\end{minipage}}

\newenvironment{beispiel}
{\begin{minipage}{\textwidth}
Beispiel
\begin{lstlisting}}
{\end{lstlisting}
\end{minipage}}

\usepackage{chngcntr,tocloft}
\counterwithin*{figure}{section}
%\counterwithin*{figure}{subsection}
%\counterwithin*{figure}{subsubsection}

\renewcommand{\thefigure}{%
  %\ifnum\value{subsection}=0
    \thesection.\arabic{figure}%
  %\else
    %\ifnum\value{subsubsection}=0
     % \thesubsection%.
   %   \arabic{figure}%
    %%\else
    %%  \thesubsubsection.\arabic{figure}%
   % \fi
 % \fi
}


% diese Festlegungen werden für die \setlemma und \setdef benötigt
\newtheorem{theorem}{Theorem}[section]
\newtheorem{lemma}[theorem]{Lemma}
\newenvironment{definition}[1][Definition]{\begin{trivlist}
\item[\hskip \labelsep {\bfseries #1}]}{\end{trivlist}}
      
% zum erzeugen einer Lemmas oder einer Defintion      
\newcommand{\setdef}[2]{\begin{definition}[#1] #2 \end{definition}}
\newcommand{\setlemma}[2]{\begin{lemma}[#1] #2 \end{lemma}}

\newcommand{\abb}[1]{Abbildung \ref{#1}}

% Quellenangaben werden mit diesen Befehlen realisiert
\newcommand{\zitat}[1]{\tiny{\cite{#1}}\normalsize}
\newcommand{\Zzitat}[1]{\tiny{#1}\normalsize}

% zum erzeugen von "todo" Zeilen
\newcommand{\blau}[1]{\textcolor{blue}{#1}}
\newcommand{\todo}[1]{\colorbox{yellow}{\parbox{\textwidth}{Todo: \textbf{#1}}}}

\newenvironment{Bilder}
  {\par\raggedbottom\null\noindent\minipage{\textwidth}\centering}
  {\endminipage\vspace{0.7cm}}
  
\newcommand{\bild}[3]{
\begin{Bilder}
 \fbox{\includegraphics[width=\textwidth]{Images/#1}}
	\captionof{figure}{#2}
	\label{#3}
	\end{Bilder}
}

\newcommand{\wbild}[4]{%\textwidth
\begin{wrapfigure}{r}{#4\textwidth}
\centering
\fbox{\includegraphics[scale=.75]{Images/#1}}
	\captionof{figure}{#2}
	\label{#3}
\end{wrapfigure}
}

\newcommand{\dbild}[3]{%\textwidth
\begin{Bilder}
 \fbox{\includegraphics[scale=.5]{Images/#1}}
	%\captionof{figure}{#2}
	\label{#3}
	\end{Bilder}
}

\newcommand{\tbild}[1]{%\textwidth
\begin{Bilder}
 \includegraphics[scale=.625]{Images/#1}
	%\captionof{figure}{#2}
	%\label{#3}
	\end{Bilder}
}

\newcommand{\gbild}[3]{%
\begin{Bilder}
 \fbox{\includegraphics[width = \textwidth]{Images/#1}}
	\captionof{figure}{#2}
	\label{#3}
	\end{Bilder}
}

\newcommand{\pbild}[3]{%\textwidth
\begin{Bilder}
 \fbox{\includegraphics[scale=.50]{Images/#1}}
	\captionof{figure}{#2}
	\label{#3}
	\end{Bilder}
}

% Ich habe öfters den Fall gehabt, dass ich gerne ein Lemma und
% ein passendes Bild daneben darstellen wollte, darum habe ich
% mir diese Gleitumgebung eingerichtet.
\newcommand{\nebeneinander}[2]{
\begin{figure}[H]
  \begin{minipage}[b]{.5\linewidth}
    \centering
#1
  \end{minipage}
  \begin{minipage}[b]{.5\linewidth}
    \centering
#2
  \end{minipage}\hfill
\end{figure}
}


% hier beginnt unser Dokument
\begin{document}
Die Kontrolleurzuweisung kann über die entsprechende Schaltfläche, einer Übungsserie, in Ihrer Veranstaltungsübersicht, erreicht werden. Damit eine Einsendung durch einen Kontrolleur bewertet werden kann, muss diese ihm hier zugeordnet werden.
\tbild{pathA}

%\cleardoublepage
% ab hier nutze ich die oben definierten Kopf- und Fusszeilen
\pagestyle{seite}
\section{AUTOMATISCHE ZUWEISUNG}
Bisher nicht zugewiesene Einsendungen (\blau{unzugeordnet}) können an Kontrolleure gleichmäßig zugewiesen werden. Dabei werden diese Studentenweise auf die Kontrolleure verteilt.

\subsubsection*{1. Kontrolleure auswählen}
Wählen Sie einen oder mehrere Kontrolleure aus.
\tbild{autoA.png}
\subsubsection*{2. Zuweisung auslösen}
\tbild{autoB.png}
Die Zuweisung war erfolgreich, wenn ein grüner Schriftzug erscheint. Sollte es eine Zeitüberschreitung der Anfrage oder andere Fehlermeldung geben, sollten Sie zunächst versuchen die Aktion zu wiederholen (eventuell wurden dabei bereits Teile der Einsendungen den Kontrolleuren zugeordnet).
\tbild{autoC.png}
Nun sollten die zugeordneten Einsendungen im Bereich \blau{MANUELLE ZUWEISUNG}, bei den entsprechenden Kontrolleuren, sichtbar sein.

\newpage
\section{MANUELLE ZUWEISUNG}
\subsection{Kontrolleursfelder}
Jeder Admin/Tutor der Veranstaltung, sofern ihm eine Einsendung zur Kontrolle zugeordnet wurde, besitzt hier ein solches Feld.
\tbild{manF.png}

\begin{enumerate}
\item Name des Kontrolleurs
\item Nutzername
\item Status in dieser Veranstaltung
\item Anzahl der zugeordneten Einsendungen (\blau{Korrekturauftrag})
\item eine zugeordnete Einsendung mit \blau{Aufgabe (Vorname Nachname,Nutzername)}
\item ein Korrekturvorschlag (dieser Nutzer wurde von diesem Kontrolleur in der letzten Serie bereits kontrolliert) mit \blau{Aufgabe (Vorname Nachname,Nutzername)}
\item hebt alle Selektionen, in diesem Feld, auf (führt keine Aktion aus)
\item selektiert alle Einsendungen, in diesem Feld (führt keine Aktion aus)
 \item selektiert nur Vorschläge, in diesem Feld (führt keine Aktion aus)
\end{enumerate}


\subsection{Zuordnung ändern}
\subsubsection*{1. Einsendungen und Zielkontrolleur auswählen}
Sie können beliebig viele Einsendungen auswählen und einen einzelnen Kontrolleur, dem sie zugeordnet werden sollen. Dabei werden bisherige Bewertungen und Korrekturen, dieser Einsendungen, gelöscht (sofern solche existieren). 
\tbild{manA.png}
\subsubsection*{2. Zuweisung auslösen}
\tbild{manB.png}
Die Zuweisung war erfolgreich, wenn ein grüner Schriftzug erscheint. Sollte es eine Zeitüberschreitung der Anfrage oder andere Fehlermeldung geben, sollten Sie zunächst versuchen die Aktion zu wiederholen (eventuell wurden dabei bereits Teile der Einsendungen den Kontrolleuren zugeordnet).
\tbild{manC.png}

\subsection{Vorschläge zuweisen}
Die Kontrolleurzuweisung wird mögliche Zuordnungen vorschlagen, weil bestimmte Studenten einem Kontrolleur bereits in der vorangegangenen Serie zugeordnet waren (diese werden \blau{hellgrau} eingefärbt).

Diese Vorschläge können (sofern angeboten) direkt übernommen werden.

\newpage
\subsubsection*{1. Zuweisung auslösen}
\tbild{manG.png}

Die Zuweisung war erfolgreich, wenn ein grüner Schriftzug erscheint. Sollte es eine Zeitüberschreitung der Anfrage oder andere Fehlermeldung geben, sollten Sie zunächst versuchen die Aktion zu wiederholen (eventuell wurden dabei bereits Teile der Einsendungen den Kontrolleuren zugeordnet).
\tbild{manC.png}

\subsection{besondere Kontrolleure}
\subsubsection*{Unzugeordnet}
\tbild{manD}
Sollte es noch unzugeordnete Einsendungen geben, werden diese hier aufgelistet. Diesem Kontrolleur können auch manuell Einsendungen zugeordnet werden.

\newpage
\subsubsection*{unbekannter Kontrolleur}
\tbild{manE}
Wenn eine Korrektur durch einen Kontrolleur erfolgte, welcher nicht mehr zu dieser Veranstaltung gehört, werden dessen Korrekturen hier aufgelistet. Zudem werden automatische Korrekturen ebenfalls hier gelistet. Dem unbekannten Kontrolleur können keine Einsendungen zugeordnet werden.

\newpage
\section{EINSENDUNGEN ERZEUGEN}
Es kann vorkommen, dass Punkte auch für Aufgaben vergeben werden müssen zu denen die Studenten aber nichts einsenden können (Präsenzaufgaben). Dann kann hier dummy-Einsendungen erzeugt werden, sodass sie einem Kontrolleur zugeordnet werden können.
\subsubsection*{1. Aktion anfragen}
\tbild{makeA.png}

\subsubsection*{2. Aktion auslösen}
Sie müssen hier den Vorgang bestätigen.
\tbild{makeB.png}

Das Erzeugen war erfolgreich, wenn ein grüner Schriftzug erscheint. Sollte es eine Zeitüberschreitung der Anfrage oder andere Fehlermeldung geben, sollten Sie zunächst versuchen die Aktion zu wiederholen (eventuell wurden bereits einige der Einsendungen erzeugt).
\tbild{makeC.png}

\newpage
\section{ZUWEISUNG AUFHEBEN}
Alle Zuordnungen und Korrekturen können hier zurückgesetzt werden.
\subsubsection*{1. Aktion anfragen}
\tbild{removeA.png}

\subsubsection*{2. Aktion auslösen}
Sie müssen hier den Vorgang bestätigen.
\tbild{removeB.png}

Das Entfernen war erfolgreich, wenn ein grüner Schriftzug erscheint. Sollte es eine Zeitüberschreitung der Anfrage oder andere Fehlermeldung geben, sollten Sie zunächst versuchen die Aktion zu wiederholen (eventuell wurden bereits einige der Zuordnungen entfernt).
\tbild{removeC.png}
\end{document}